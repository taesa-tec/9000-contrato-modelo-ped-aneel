%-----------------------------------------------------------
% Contrato de P\&D TAESA
% Version 1.0 (Fevereiro 2020)
%
% Esta minuta de contrato foi elaborada para avaliar a facilidade de de edição de contratos por meio do uso do LaTeX, permitindo a utilização de ferramentas de colaboração e controle de versão.
%
%-----------------------------------------------------------

%-----------------------------------------------------------
%	Configuração Inicial dos pacotes e funcionalidades 
%   Vide arquivo 10_Configuracao_inicial.tex
%-----------------------------------------------------------
%%%------------------------------------------------------------------------------%%
%%%	PAGE LAYOUT SPECIFICATIONS
%%%------------------------------------------------------------------------------%%
%\documentclass[a4paper,12pt,twoside]{report} % Página default, 2Lados, font size is 12pt on A4 paper
\documentclass[a4paper,12pt,notitlepage]{report} % Página default, single-side, font size is 12pt on A4 paper
\usepackage{geometry} % Required to modify the page layout
\geometry{
         headheight=3cm,
         top=4.0cm,
         bottom=5.0cm,
         left=2.0cm,
         right=2.0cm}

%%%%%% ==============================  PACOTES  ============================== %%%%%%
%%% --- Listas e enumeradores ---%
\usepackage{enumitem}
\usepackage[sharp]{easylist}

%%%%  --- Layout de página ---%
\usepackage{fancyhdr}  % customized pages style

%%%  --- Gerador de Loren Ipsum ---%
\usepackage{lipsum}% juts to generate text for the example

%%%  --- paginação referencial "Pagina X de Y"  ---%
\usepackage{lastpage} 

%%%  --- Permite caracteres acentuados ---%
\usepackage[utf8]{inputenc} 

%%%%  --- Fontes  ---%
%\usepackage{newtxmath} % 
%\usepackage{mathptmx} % Use the Adobe Times Roman as the default text font together with math symbols from the Sym­bol, Chancery and Com­puter Modern fonts
%\usepackage{newtxtext} % 
%\usepackage{amsfonts} % Use de AMS font
\usepackage{avant} % Use the Avantgarde font 
\usepackage{microtype}
\usepackage{helvet}
\usepackage[normalem]{ulem}
\usepackage{soul}

%%%%  --- Tabelas ---%
%\usepackage{tabularx} % in the preamble
%\usepackage{showframe}% http://ctan.org/pkg/showframe
%\usepackage{booktabs}% http://ctan.org/pkg/booktabs
\usepackage{array}
\usepackage{graphicx}

%%%%  --- PDF Features  ---%
%\usepackage[author={P&D TAESA}]{pdfcomment}


%%%%%% ==============================  Configurações  ============================== %%%%%%

%%%--------------------------------------------------------------------------------
%%%	FONT SPECIFICATIONS
%%%--------------------------------------------------------------------------------

%Escolha da Fonte
\renewcommand{\familydefault}{\sfdefault}
%%%%\renewcommand{\rmdefault}{ptm} %não sei para que serve
    %\rmdefault %selects a roman (i.e., serifed) font family
    %\sfdefaultand %selects a sans serif font family
    %\ttdefault %selects a monospaced (“typewriter”) font family 

    %\rmfamily %selects a roman (i.e., serifed) font family
    %\sffamily %selects a sans serif font family
    %\ttfamily %selects a monospaced (“typewriter”) font family 
        %Within each font family, the following declarations select the “series” (i.e., dark-ness or stroke width),
        %\mdseries %regular
        %\bfseries %bold
        %and the “shape” (i.e., the form of the letters)
        %\upshapeupright
        %\slshapeslanted
        %\itshapeitalic
        %\scshapeCapsandSmallCap

%%% Configuração fina
%%% http://tug.ctan.org/tex-archive/macros/latex/contrib/microtype/microtype.pdf
%%\microtypecontext{
%%    activate={true,nocompatibility},%%% activate={true,nocompatibility} - activate protrusion and expansion
%%    final, %%% final - enable microtype; use "draft" to disable
%%    tracking=true,%%% tracking=true, kerning=true, spacing=true - activate these techniques
%%    protrusion=true,
%%    expansion,
%%    kerning=true,
%%    spacing=true,
%%    factor=1100,%%% factor=1100 - add 10% to the protrusion amount (default is 1000)
%%    stretch=10, %%% stretch=10, shrink=10 - reduce stretchability/shrinkability (default is 20/20)
%%    shrink=10,
%%    spacing=nonfrench
%%    }
%%

%%%----------------------------------------------------------------------------------------
%%%	Especificações do Sumário 
%%%----------------------------------------------------------------------------------------
\setcounter{tocdepth}{1} % Profundidade do sumário
\renewcommand{\contentsname}{SUMÁRIO}


%%%----------------------------------------------------------------------------------------
%%%	PARAGRAPH SPACING SPECIFICATIONS
%%%----------------------------------------------------------------------------------------
\setlength{\parindent}{15mm} % Retira indentação dos parágrafos
\setlength{\parskip}{2.5mm} % Whitespace between paragraphs
\usepackage{indentfirst}


%%%----------------------------------------------------------------------------------------
%%%	SECTION TITLE SPECIFICATIONS
%%%----------------------------------------------------------------------------------------

\usepackage{titlesec} % Required for modifying section titles

%%% Número e nome dos capítulos na mesma linha
\titleformat{\chapter} [hang]
    {\sffamily\large\bfseries} % Title font customizations
    {\chaptertitlename\ \thechapter.} % chapter number
    {5pt} % Whitespace between the number and title
    {\large} % Title font size
    \titlespacing*{\chapter}{0mm}{5mm}{0mm} % Left, top and bottom spacing around the title
    
\titleformat{\section} %[block] % Customize the \section{} section title
    {\sffamily\large\bfseries} % Title font customizations
    {\thesection} % Section number
    {5pt} % Whitespace between the number and title
    {\large} % Title font size
    \titlespacing*{\section}{0mm}{5mm}{0mm} % Left, top and bottom spacing around the title

\titleformat{\subsection} % Customize the \subsection{} section title
    {\sffamily\large\bfseries} % Title font customizations
    {\thesubsection} % Subsection number
    {5pt} % Whitespace between the number and title
    {\large} % Title font size
    \titlespacing*{\subsection}{0mm}{5mm}{0mm} % Left, top and bottom spacing around the title

\titleformat{\subsubsection} % Customize the \subsection{} section title
    {\sffamily\large\bfseries} % Title font customizations
    {\thesubsubsection} % Subsection number
    {5pt} % Whitespace between the number and title
    {\large} % Title font size
    \titlespacing*{\subsubsection}{0mm}{5mm}{0mm} % Left, top and bottom spacing around the title

\renewcommand{\chaptername}{CLÁUSULA} % Altera nomes dos capítulos de "Chapter" para "clausula"
%%%\renewcommand{\thechapter}{\Roman{chapter}} %Altera numeração de capítulos para números romanos

%%%% 2 capitulos na mesma pagina
\usepackage{etoolbox}
    \makeatletter
    \patchcmd{\chapter}{\if@openright\cleardoublepage\else\clearpage\fi}{}{}{}
    \makeatother


%%%----------------------------------------------------------------------------------------
%%%	Especificações das listas e numerações 
%%%----------------------------------------------------------------------------------------
%\setcounter{secnumdepth}{9} % Profundidade da numeração


%\renewcommand*{\theenumi}{\thesection.\arabic{enumi}} % Permite numeração continuada a partir dos capítulos e seções.
%\renewcommand*{\theenumii}{\theenumi.\arabic{enumii}} % Permite numeração continuada a partir dos capítulos e seções.
%\usepackage{remreset} % OBSOLETE Para numeração de seções continuamente, independente dos capítulos

%\makeatletter 
%  \@removefromreset{section}{chapter}
%\makeatother


%%%-------COMENTARIOS---------------------------------------------------------------------------------
\iffalse
teste
\fi


%%%-----------------------------------------------------------
%%%	Cabeçalho e Rodapé
%%%-----------------------------------------------------------
%% Redefine the plain page style
\fancypagestyle{plain}{%
    \fancyhf{}
    \renewcommand{\headrulewidth}{0.4pt}
        \fancyfoot[L]{}    
        \fancyfoot[C]{}
        \fancyhead[R]{\sffamily{PROJETO P\&D: \ApelidoProjeto\ -\ Contrato nº \NumeroContrato}}
    \renewcommand{\footrulewidth}{0.4pt}
        \pagenumbering{arabic}
        \fancyfoot[L]{}
        \fancyfoot[C]{\fontsize{9}{9} \sffamily{Chancelas e Rubricas:}}
        \fancyfoot[R]{\fontsize{9}{9} \sffamily {Página\ \thepage\ de \pageref{LastPage}}}
        
}

%% Redefine the empty  page style
\fancypagestyle{empty}{%
    \fancyhf{}
    \renewcommand{\headrulewidth}{0.4pt}
        \fancyfoot[C]{}
        \fancyhead[R]{}
    \renewcommand{\footrulewidth}{0.4pt}
        \pagenumbering{arabic}
        \fancyfoot[L]{\fontsize{9}{9} \sffamily{Chancelas e Rubricas:}}
        \fancyfoot[C]{}
        \fancyfoot[R]{\fontsize{9}{9} \sffamily{Página\ \thepage\ de \pageref{LastPage}}}
} % Input the 10_Configuracao_inicial.tex file which specifies the document packages and features

%-----------------------------------------------------------
%	PREENCHIMENTO DE DADOS
%-----------------------------------------------------------

% Dados do Projeto
\newcommand{\NumeroProjeto}{0045}
\newcommand{\CodAneelProjeto}{PD-0XXXX-XXXX/201X}
\newcommand{\ApelidoProjeto}{\NumeroProjeto-TituloProjeto}

\newcommand{\TituloCompletoProjetoDE}{título completo do projeto do projeto}
\newcommand{\ObjetoDoProjetoDE}{Metodologia para Mitigacao de Riscos de Desligamento e Avaliacao de Metodos de Manutencao de Faixas de Servidao}
\newcommand{\ObjetivoResumidoDoProjetoDE}{descrever o objetivo do projeto nesta etapa da cadeia de inovação}

\newcommand{\TituloCompletoProjetoCS}{título completo do projeto do projeto}
\newcommand{\ObjetoDoProjetoCS}{Metodologia para Mitigacao de Riscos de Desligamento e Avaliacao de Metodos de Manutencao de Faixas de Servidao}
\newcommand{\ObjetivoResumidoDoProjetoCS}{descrever o objetivo do projeto nesta etapa da cadeia de inovação}

\newcommand{\TituloCompletoProjetoIM}{título completo do projeto do projeto}
\newcommand{\ObjetoDoProjetoIM}{Metodologia para Mitigacao de Riscos de Desligamento e Avaliacao de Metodos de Manutencao de Faixas de Servidao}
\newcommand{\ObjetivoResumidoDoProjetoIM}{descrever o objetivo do projeto nesta etapa da cadeia de inovação}


% Dados de Suprimentos
\newcommand{\NumeroContrato}{10000}
\newcommand{\DataDaCelebracao}{01/01/2019}

% Dados da Contratante
\newcommand{\NomeContratante}{TRANSMISSORA ALIANÇA DE ENERGIA ELÉTRICA S.A.}
\newcommand{\NomeContratanteResumido}{TAESA}
\newcommand{\EnderecoContratante}{Praça XV de Novembro}
\newcommand{\numeroEnderecoContratante}{20}
\newcommand{\ComplementoEnderecoContratante}{salas 601 e 602}
\newcommand{\BairroContratante}{Centro}
\newcommand{\CEPContratante}{20010-010}
\newcommand{\CidadeEnderecoContratante}{Rio de Janeiro}
\newcommand{\EstadoEnderecoContratante}{Rio de Janeiro}
\newcommand{\NumCNPJContratante}{07.859.971/0001-30}

\newcommand{\DiretorUmContratante}{Helvécio Miranda Magalhães Júnior}
\newcommand{\DiretorDoisContratante}{João da Silva}


% Dados da NomeExecutora A
\newcommand{\NomeExecutoraA}{Empresa XPTO}
\newcommand{\NomeExecutoraAResumido}{A.C.M.E}
\newcommand{\EnderecoExecutoraA}{Rua Alogoana}
\newcommand{\numeroEnderecoExecutoraA}{10000}
\newcommand{\ComplementoEnderecoExecutoraA}{5º andar}
\newcommand{\BairroExecutoraA}{Centro}
\newcommand{\CEPExecutoraA}{12345-123}
\newcommand{\CidadeEnderecoExecutoraA}{São Paulo}
\newcommand{\EstadoEnderecoExecutoraA}{São Paulo}
\newcommand{\NumCNPJExecutoraA}{00.000.000/0000-00}

\newcommand{\DiretorUmExecutoraA}{José da Silva}
\newcommand{\DiretorDoisExecutoraA}{Antônio da Silva}


% Dados da Executora B
\newcommand{\NomeExecutoraB}{Empresa XPTO2}
\newcommand{\NomeExecutoraBResumido}{A.C.M.E2}
\newcommand{\EnderecoExecutoraB}{Rua Alogoana2}
\newcommand{\numeroEnderecoExecutoraB}{100002}
\newcommand{\ComplementoEnderecoExecutoraB}{2º andar}
\newcommand{\BairroExecutoraB}{Centro2}
\newcommand{\CEPExecutoraB}{22345-123}
\newcommand{\CidadeEnderecoExecutoraB}{São Paulo2}
\newcommand{\EstadoEnderecoExecutoraB}{São Paulo2}
\newcommand{\NumCNPJExecutoraB}{20.000.000/0000-00}

\newcommand{\DiretorUmExecutoraB}{José da Silva}
\newcommand{\DiretorDoisExecutoraB}{Antônio da Silva}


% Dados Interveniente A 
\newcommand{\IntervenienteANome}{[Interveniente]}
\newcommand{\IntervenienteAEndereco}{\ContratanteEndereco}
\newcommand{\IntervenienteACNPJ}{\textbf{[$\bullet$]}}

\newcommand{\RepresentanteAIntervenienteA}{\textbf{[$\bullet$]}} %Representante A
\newcommand{\RepresentanteAIntervenienteACI}{\textbf{[$\bullet$]}}
\newcommand{\RepresentanteAIntervenienteACPF}{\textbf{[$\bullet$]}}

\newcommand{\RepresentanteBIntervenienteA}{\textbf{[$\bullet$]}} %Representante B
\newcommand{\RepresentanteBIntervenienteACI}{\textbf{[$\bullet$]}}
\newcommand{\RepresentanteBIntervenienteACPF}{\textbf{[$\bullet$]}}

% Dados Interveniente B 
\newcommand{\IntervenienteBNome}{[Interveniente]}
\newcommand{\IntervenienteBEndereco}{\ContratanteEndereco}
\newcommand{\IntervenienteBCNPJ}{\textbf{[$\bullet$]}}

\newcommand{\RepresentanteAIntervenienteB}{\textbf{[$\bullet$]}} %Representante A
\newcommand{\RepresentanteAIntervenienteBCI}{\textbf{[$\bullet$]}}
\newcommand{\RepresentanteAIntervenienteBCPF}{\textbf{[$\bullet$]}}

\newcommand{\RepresentanteBIntervenienteB}{\textbf{[$\bullet$]}} %Representante B
\newcommand{\RepresentanteBIntervenienteBCI}{\textbf{[$\bullet$]}}
\newcommand{\RepresentanteBIntervenienteBCPF}{\textbf{[$\bullet$]}}

% Valor do Contrato e Dotação Orçamentária
\newcommand{\ValorContrato}{[$\bullet$]}
\newcommand{\ValorTotalCP}{[$\bullet$]}
\newcommand{\DotacaoOrcamentaria}{$\bullet$]}
 % Input the 02_Dados_do_Contrato.tex file which specifies the document layout and style

%%%%%% ===========================  Inicio do Contrato  ============================ %%%%%%
\begin{document}
\thispagestyle{empty}
\noindent
\begin{center}
    \begin{tabular}{ | p{170mm} | }
      \hline
      \\ [0.07cm]
      {\sffamily\large\bfseries CONTRATO DE PESQUISA E DESENVOLVIMENTO} \\ [0.5cm]
      {\sffamily\large\bfseries \ApelidoProjeto} \\ [0.5cm]
      {\sffamily Nº do CONTRATO: \NumeroContrato} \\ [0.5cm]
      {\sffamily DATA DA CELEBRAÇÃO: \DataDaCelebracao} \\ [0.5cm]
      {\sffamily OBJETO: \ObjetoDoProjetoDE} \\ [0.5cm]
      \\ [0.0cm]
      \hline
    \end{tabular}
  \end{center} % Input the 03_Cabecalho.tex file
\pagestyle{plain} % define o estilo do cabeçalho e rodaé da pagina, com linhas e numeração ao centro

%%%%%% ===========================  Corpo do Contrato  ============================ %%%%%%

%\lipsum{1-20}

Pelo presente instrumento, e na melhor forma de direito, as sociedades:

Na qualidade de \textbf{PROPONENTE}: 

A \textbf{\NomeContratante} (“\textbf{\NomeContratanteResumido}”), com sede no município \CidadeEnderecoContratante e estado do(e) \EstadoEnderecoContratante, na \EnderecoContratante, nº \numeroEnderecoContratante, \ComplementoEnderecoContratante, \BairroContratante, CEP \CEPContratante, inscrita no CNPJ/MF sob o nº \NumCNPJContratante, representada na forma de seu Estatuto Social;

Na qualidade de \textbf{EXECUTORA(S)}:	

A \textbf{\NomeExecutoraA}, com sede na \EnderecoExecutoraA, nº \numeroEnderecoExecutoraA, \ComplementoEnderecoExecutoraA, \BairroExecutoraA, CEP \CEPExecutoraA, \CidadeEnderecoExecutoraA. – \EstadoEnderecoExecutoraA, devidamente inscrita no CNPJ/MF sob nº \NumCNPJExecutoraA, neste ato devidamente representada nos termos de seu Estatuto Social, doravante denominada como  \textbf{\NomeExecutoraAResumido}; e.

A \textbf{\NomeExecutoraB}, com sede na \EnderecoExecutoraB, nº \numeroEnderecoExecutoraB, \ComplementoEnderecoExecutoraB, \BairroExecutoraB, CEP \CEPExecutoraB, \CidadeEnderecoExecutoraB. – \EstadoEnderecoExecutoraB, devidamente inscrita no CNPJ/MF sob nº \NumCNPJExecutoraB, neste ato devidamente representada nos termos de seu Estatuto Social, doravante denominada como  \textbf{\NomeExecutoraBResumido}; e.

Sendo \textbf{PROPONENTE} e \textbf{EXECUTORA(S)}, em conjunto, denominadas \textbf{PARTÍCIPES};

Têm entre si, justo e acordado, celebrar o presente instrumento, adiante denominado simplesmente \textbf{CONTRATO}, que se regerá pelas seguintes cláusulas e condições:

\chapter{OBJETO E DOCUMENTAÇÃO INTEGRANTE}
\begin{easylist}
    \ListProperties(Start1*=\thechapter,Margin2=0cm,Margin3=1cm,Margin4=2.3cm,Margin5=3.9cm)
    ## O presente \textbf{CONTRATO} tem por objeto a execução, pela(s) \textbf{EXECUTORA(S)}, de um projeto de Pesquisa e Desenvolvimento (“\textbf{P\&D}”), para cada uma das seguintes \uline{etapas da cadeia de inovação}: i)Desenvolvimento Experimental (DE), ii) Cabeça de Série  e Lote Pioneiro (CS/LP) e iii)Inserção no Mercado (IM), sendo imperativa a entrega pela(s) EXECUTORA(S), ao final de cada projeto, os produtos desenvolvidos em estágio finalizado, sendo desde já, não acatada qualquer justificativa de imprevisibilidade inerente à pesquisa básica e/ou dirigida.
		### O projeto inicial, relativo a etapa da cadeia de inovação i)Desenvolvimento Experimental (DE),  será identificado pelo código ANEEL \CodAneelProjeto, intitulado “\TituloCompletoProjetoDE”, adiante denominado simplesmente “PROJETO D.E.”, cujo objetivo é \ObjetivoResumidoDoProjetoDE.
        ### O segundo projeto, relativo as etapas da cadeia de inovação ii) Cabeça de Série  e Lote Pioneiro (CS/LP), será identificado pelo título “\TituloCompletoProjetoCS”, adiante denominado simplesmente “PROJETO CS/LP”, cujo objetivo é \ObjetivoResumidoDoProjetoCS.
        ### O terceiro e último projeto, relativo as etapas da cadeia de inovação iii)Inserção no Mercado (IM),  será identificado pelo título “\TituloCompletoProjetoIM”, adiante denominado simplesmente “PROJETO I.M.”, cujo objetivo é \ObjetivoResumidoDoProjetoIM.
		### Em conjunto, são adiante denominados simplesmente “PROJETO”
	## Nos termos deste CONTRATO, a(s) EXECUTORA(S) se obrigam a prestar todos os serviços e a fornecer todo o ferramental, documentação e materiais necessários à perfeita execução do objeto contratual, em conformidade com seus anexos e com todas as normas aplicáveis ao objeto deste CONTRATO, em especial o pleno atendimento a todos os módulos dos Procedimentos do Programa de Pesquisa e Desenvolvimento, adiante denominado simplesmente PROP\&D, aprovado pela Resolução Normativa nº 754/2016 e pela Lei 9.991 de 24 de julho de 2000.
	## A(s) EXECUTORA(S) declaram que o objeto do PROJETO está enquadrado entre os que compõem o seu objeto social e que detêm conhecimento e experiência na execução dos mesmos, bem como que possui todos os registros e licenças necessários para sua realização, inexistindo qualquer restrição ou impedimento a respeito.
	## São partes integrantes deste CONTRATO, além do seu texto principal, o Anexo I, adiante denominado simplesmente “PLANO DE TRABALHO” e o Anexo II adiante denominado simplesmente "ESPECIFICAÇÃO TÉCNICA".
	## Na hipótese de conflito entre as disposições constantes do corpo deste CONTRATO e os documentos anexos acima mencionados, o disposto no corpo deste CONTRATO deverá prevalecer. Os documentos ora anexos, por sua vez, prevalecerão, uns sobre os outros, de acordo com a ordem em que se apresentam.
	## O objeto do presente instrumento será executado, pela(s) EXECUTORA(S), nas instalações definidas pela PROPONENTE, em etapas e fases listadas no PLANO DE TRABALHO e gerando os seguintes produtos, nessa ordem:
	
	
	\noindent
\begin{table}[ht]\caption{\textbf{Desenvolvimento Experimental (DE)}}% title of Table
\begin{tabular}{|p{10mm}|p{15mm}|p{15mm}|p{120mm}|}% centered columns (4 columns)
\hline %inserts double horizontal lines
Etapa & Cadeia Inovação & Duração (meses) & Produtos \\ % inserts table %heading
\hline % inserts single horizontal line
	1&(CS)&(XX)&
\begin{enumerate} 
	\item Busca de Anterioridade emitida por empresa contratada e independente, confirmando o ineditismo de todos os produtos esperados do projeto, realizada nas seguintes bases:  Anais de eventos do setor elétrico tais como e sem se limitar a: Citenel, SENDI, SNPTEE e outros; Bancos de publicações de produção científica tais como e sem se limitar a: SCielo, Engineering Village, Scopus, etc; Bancos de patentes e registros tais como e sem se limitar a: INPI e USPTO. 
	\item Especificação detalhada do funcionamento do sistema (hardware e software) com suas funcionalidades, relacionando as normas técnicas que devem ser atendidas tanto para fabricação quanto para instalação. Essas especificações deverão atender às características, condições e obrigações de operação e manutenção, bem como a lista de testes em conformidade com as normas técnicas aplicáveis e, por fim, a metodologia de funcionamento detalhada do sistema.
  \end{enumerate} \\

% [1ex] % [1ex] adds vertical space
\hline %inserts single line
\end{tabular}
\label{table:nonlin}% is used to refer this table in the text
\end{table}

  Etapa	Cadeia Inovação	Duração (meses)	Produtos


      \end{easylist}






% http://www.bakoma-tex.com/doc/latex/easylist/easylist-doc.pdf
\chapter{OBJETO E DOCUMENTAÇÃO INTEGRANTE}
    \begin{easylist}
        \ListProperties(Start1*=\thechapter,Start2=17,Hang=true,Margin=5cm)
        # First proposition.
        ## Numbering doesn't work.
        \ListProperties(Start2=17)
        ## This is better.
        # Hey, I can't move on!
        # I must be stuck to an external counter!
        \ListProperties(Start1*=NA)
        # Okay, it works again.
        \end{easylist}









%%%%% ======================  Fim do Corpo do Contrato  ============================ %%%%%%
\end{document}